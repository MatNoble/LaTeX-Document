\documentclass[a4paper, 12pt, UTF8]{ctexart}
\usepackage{matnoble-doc-cn}
\begin{document}
\title{\bf \LaTeX 中文文档模板} \author{\bf
  \href{https://www.hustmatnoble.tk/2019/11/contact-me.html}{数系家园}}
\date{}

\maketitle
\tableofcontents
\footnote{\noindent \createtext{2019 年 12 月 5 日} \newline
  \updatetext{\today}}.

\clearpage
\listoflistings

\clearpage

\section{随机文本}
\zhlipsum[2-3]

\clearpage

\section{通常环境}

\subsection{列表}

\subsubsection{无序列表}
\begin{itemize}
    \item Sth
    \item Sth
    \item $\cdots$
\end{itemize}
\subsubsection{有序列表}
\begin{enumerate}
    \item sth
    \item sth
    \item $\cdots$
\end{enumerate}
或者
\begin{itemize}
    \item[(1)] Sth
    \item[(2)] Sth
    \item[(3)] $\cdots$
\end{itemize}

\bigskip

\noindent {\bf 使用 \autoref{mybox} 得到更漂亮列表环境 }

\medskip

\begin{mybox}{无序列表}
    \begin{itemize}[leftmargin = 10pt]
        \item Sth
        \item Sth
        \item $\cdots$
    \end{itemize}
\end{mybox}

\begin{listing}[ht]
    \begin{minted}[ frame=single, framesep=2mm, baselinestretch=1.2,
        bgcolor=MaterialBlue50, rulecolor=MaterialBlue,
        fontsize=\footnotesize, mathescape, autogobble ]{tex}
        无序
        \begin{itemize}
            \item Sth
            \item Sth
            \item $\cdots$
        \end{itemize}
        有序
        \begin{enumerate}
            \item sth
            \item sth
            \item $\cdots$
        \end{enumerate}
        带边框
        \begin{mybox}{列表}
            \begin{itemize}[leftmargin = 10pt]
                \item Sth
                \item Sth
                \item $\cdots$
            \end{itemize}
        \end{mybox}
    \end{minted}
    \caption{\em mybox}
    \label{mybox}
\end{listing}

\clearpage
\subsection{表格}
\begin{table}[ht]
    \centering
    \caption{\em 这是表格}
    \vskip 0.1in
    \label{table}
    \begin{tabular}{c|cccc}
      \hline
      \hline
      \rule{0pt}{3ex}
      序号 & 姓名 & 年龄 & 学号 & 性别 
                                  \rule[-1.2ex]{0pt}{0pt} \\\hline
      001 & *  &  *  & *  & * \\ 
      002 & *  &  *  & *  & * \\
      003 & *  &  *  & *  & * \\      
      004 & *  &  *  & *  & * \\
      \hline
      \hline 
    \end{tabular}
\end{table}

\begin{listing}[ht]
    \begin{minted}[ frame=single, framesep=2mm, baselinestretch=1.2,
        bgcolor=MaterialBlue50, rulecolor=MaterialBlue,
        fontsize=\footnotesize, mathescape, autogobble ]{tex}
        \begin{table}[ht]
            \centering
            \caption{\em 这是表格}
            \vskip 0.1in
            \label{table}
            \begin{tabular}{c|cccc}
              \hline
              \hline
              \rule{0pt}{3ex}
              序号 & 姓名 & 年龄 & 学号 & 性别  
                                          \rule[-1.2ex]{0pt}{0pt} \\\hline
              001 & *  &  *  & *  & * \\ 
              002 & *  &  *  & *  & * \\
              003 & *  &  *  & *  & * \\      
              004 & *  &  *  & *  & * \\
              \hline
              \hline 
            \end{tabular}
        \end{table}
    \end{minted}
    \caption{\em 表格}
    \label{table}
\end{listing}

\clearpage
\subsection{图}

\subsubsection{单张图}
\begin{figure}[H]
    \centering \includegraphics[width=0.15\textwidth]{google.png}
    \caption{\em 单张图}
    \label{fig:mesh}
\end{figure}

\subsubsection{两张图并列}
\begin{figure}[H]
    \centering
    \begin{subfigure}{.48\textwidth}
        \centering
        % include first image
        \includegraphics[width=.3\linewidth]{google.png}
        \caption{\em 子图 1}
        \label{fig:v21}
    \end{subfigure}
    \begin{subfigure}{.48\textwidth}
        \centering
        % include second image
        \includegraphics[width=.3\linewidth]{google.png}
        \caption{\em 子图 2}
        \label{fig:v22}
    \end{subfigure}
    \caption{\em 并列两张图}
    \label{fig:v2}
\end{figure}

\begin{listing}[ht]
    \begin{minted}[ frame=single, framesep=2mm, baselinestretch=1.2,
        bgcolor=MaterialBlue50, rulecolor=MaterialBlue,
        fontsize=\footnotesize, mathescape, autogobble ]{tex}
        \begin{figure}[H]
            \centering
            \begin{subfigure}{.48\textwidth}
                \centering
                \includegraphics[width=.5\linewidth]{google.png}
                \caption{\em 子图 1}
                \label{fig:v21}
            \end{subfigure}
            \begin{subfigure}{.48\textwidth}
                \centering
                \includegraphics[width=.5\linewidth]{google.png}
                \caption{\em 子图 2}
                \label{fig:v22}
            \end{subfigure}
            \caption{\em 并列两张图}
            \label{fig:v2}
        \end{figure}
    \end{minted}
    \caption{\em 并列图}
    \label{subfigure}
\end{listing}

\clearpage

\section{定理类环境}
\begin{defn}{}{}
    
\end{defn}

\begin{lem}{}{}
    
\end{lem}

\begin{thm}{}{}
    \begin{case}
        
    \end{case}
    \begin{case}
        
    \end{case}
\end{thm}

\begin{rem}
    
\end{rem}

\begin{cor}{}{}
    
\end{cor}

\begin{exa}
    
\end{exa}

\begin{proof}
    
\end{proof}

\clearpage
\section{数学公式}

\[
    \int x^{2} \dif x
\]

为证 $a = b$, 需要证明

{\bf 无标号}
\[
    a < b + \epsilon, \and b < a + \epsilon.
\]

{\bf 有标号}
\begin{equation}
    \label{eq:1}
    a < b + \epsilon, \and b < a + \epsilon.
\end{equation}

\vskip2em

PNP/Stokes 方程组
\begin{empheq}[left=\empheqlbrace]{align}
    & \partial_{t} - \nabla\cdot[D_{i}(\nabla C_{i} + q_{i} \nabla
    \Phi C_{i}) - \bmu C_{i}] = F_{i},
    \\[3pt]
    & -\nabla \cdot (\epsilon \nabla \Phi) = (C_{1} - C_{2}) + F_{3},
    \\[3pt]
    & \partial_{t}\bmu - \Delta \bmu + \nabla p = -(C_{1} -
    C_{2})\nabla \Phi + F_{4},
    \\[3pt]
    & \nabla \cdot \bmu = 0.
\end{empheq}

矩阵
\[
    \begin{bmatrix}
        2 & -1 & 0\\
        -1& 2  &-1\\
        0 & -1 & 2
    \end{bmatrix}
\]

\clearpage
\begin{thebibliography}{99}
    \bibitem{1}
    \bibitem{2}
    \bibitem{3}
\end{thebibliography}

\clearpage

\begin{appendices}
    \section{代码}
    \begin{listing}[ht]
        \begin{minted}[ frame=single, framesep=2mm,
            baselinestretch=1.2, bgcolor=MaterialBlue50,
            rulecolor=MaterialBlue, fontsize=\footnotesize, linenos,
            mathescape, autogobble ]{python}
            import matplotlib.pyplot as plt import numpy as np
        
            plt.figure(num = 1, figsize=(8, 6)) n =
            np.linspace(1,100,100) plt.plot(n, 1/n, 'bx')
            plt.xlabel(r'$ n $') plt.ylabel(r'$ \frac{1}{n} $')
        
            plt.figure(num = 2, figsize=(8, 6)) n =
            np.linspace(1,100,100) plt.plot(n, np.sin(n)/n, 'bx')
            plt.xlabel(r'$ n $') plt.ylabel(r'$ \frac{\sin(n)}{n} $')
        
            plt.show()
        \end{minted}
        \caption{\em Python}
    \end{listing}

    \begin{listing}[ht]
        \begin{minted}[ frame=single, framesep=2mm,
            baselinestretch=1.2, bgcolor=MaterialBlue50,
            rulecolor=MaterialBlue, fontsize=\footnotesize, linenos,
            mathescape, autogobble, breaklines ]{matlab}
            figure() plot(XX,YY,'k-'),hold on plot(XX',YY','k-'), hold
            on B= plot(boundary(3,:), boundary(4,:), 'b.',
            'markersize', 25); hold on I = plot(index(:,1),
            index(:,2), 'r.', 'markersize',25); hold off axis equal
            set(gca,'xtick',[],'ytick',[]) xlim(X) ylim(Y)
            set(gca,'looseInset',[0 0.01 0 0.01]) h = legend([B, I],
            'boundary nodes', 'inside nodes',
            'Location','bestoutside'); set(h, 'Fontsize', 10)
        \end{minted}
        \caption{\em Matlab}
    \end{listing}
\end{appendices}

\end{document}
